    \chapter{JADWAL PENELITIAN}

% Ubah tabel berikut sesuai dengan isi dari rencana kerja
\newcommand{\w}{}
\newcommand{\G}{\cellcolor{gray}}
\newcommand{\B}{\cellcolor{blue}}
\begin{table}[H]
  \captionof{table}{Tabel timeline}
  \label{tbl:timeline}
  \begin{tabular}{|p{3.5cm}|c|c|c|c|c|c|c|c|c|c|c|c|c|c|c|c|}

    \hline
    \multirow{2}{*}{Kegiatan} & \multicolumn{16}{|c|}{Minggu}                                                                       \\
    \cline{2-17}              &
    1                         & 2                             & 3  & 4  & 5  & 6  & 7  & 8  & 9  & 10 & 11 & 12 & 13 & 14 & 15 & 16 \\
    \hline

    % Gunakan \G untuk mengisi sel dan \w untuk mengosongkan sel
    Studi literasi          &
    \G                        & \G                            & \G & \G & \w & \w & \w & \w & \w & \w & \w & \w & \w & \w & \w & \w \\
    \hline

    Perancangan dan desain sistem kendali           &
    \w                        & \w                            & \w & \w & \G & \G & \G & \G & \w & \w & \w & \w & \w & \w & \w & \w \\
    \hline

    Perakitan dan pemasangan pada kursi roda              &
    \w                        & \w                            & \w & \w & \w & \w & \w & \w & \G & \G & \G & \G & \w & \w & \w & \w \\
    \hline

    Pengujian alat dan evaluasi hasil       &
    \w                        & \w                            & \w & \w & \w & \w & \w & \w & \w & \w & \w & \w & \G & \G & \G & \G \\
    \hline
  \end{tabular}
\end{table}

Pada \emph{timeline} yang tertera di Tabel \ref{tbl:timeline}, pada empat minggu pertama dilakukan studi literasi terdahulu mengenai apa saja yang ada di dasar teori. Setelah melakukan studi literasi, dilakukan perancangan dan desain sistem kendali. mulai dari bagaimana pemasangan \textit{raspberry pi} pada kursi roda, perkabelan rangkaian, dan peletakan \textit{touchpad} pada kursi roda. Setelah merancang dan mendesain semua dengan baik, dimulailah perakitan dan pemasangan semua alat kepada kursi roda. yang terakhir adalah melakukan uji coba alat dan mengevaluasi hasil dari uji coba alat tersebut. 

