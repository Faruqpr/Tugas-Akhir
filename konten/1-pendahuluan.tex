\chapter{PENDAHULUAN}

\section{Latar Belakang}

Perkembangan teknologi di masa sekarang sudah jauh berkembang sehingga dapat membantu bidang lain, salah satunya dalam bidang kesehatan. Teknologi dapat membantu penggunanya seperti dokter atau pun pasien. Perkembangan yang terjadi di bidang kesehatan salah satunya adalah pengembangan teknologi pada kursi roda. Menurut Organisasi Perjalanan Dunia Perserikatan Bangsa-Bangsa (UNWTO), jumlah orang dengan disabilitas meningkat 2\% per tahun sejak 2001 dan diperkirakan mencapai sekitar 26,8 juta \parencite{GujjarSmartWheelchair}. Kursi roda menjadi salah satu pilihan alternatif untuk membantu kegiatan sehari-hari. Pengembangan teknologi pada kursi roda dapat membantu pasien dalam mengoperasikan kursi roda tersebut. 

Kursi roda adalah alat bantu mobilitas yang dibuat khusus untuk membantu orang yang memiliki keterbatasan dalam berjalan atau berdiri akibat cedera, penyakit, atau kondisi lainnya. Kursi roda memberikan penggunanya kebebasan dan kemandirian untuk bergerak dan berinteraksi dengan sekitarnya. Perkembangan teknologi pada kursi roda dapat membantu penggunanya untuk bergerak sendiri tanpa bantuan dari orang lain, salah satu alat kemudi yang bisa dikembangkan bisa dalam bentuk touchpad. Penggunaan touchpad memungkinkan penggunanya untuk menggunakan Gerakan otot yang lebih sedikit dan tekanan otot yang lebih rendah \parencite{Salafi2009TouchScreen}.

Touchpad atau dikenal juga dengan trackpad merupakan salah satu alat penunjuk yang biasa digunakan pada laptop. Ini digunakan untuk menggantikan mouse sebagai metode berinteraksi dengan antarmuka pengguna grafis (GUI) di komputer. Touchpad ini memungkinkan pengguna untuk mendefinisikan gerakan jari sebagai pintasan untuk meningkatkan produktivitas dan efisiensi kerja dan hal ini sangat membantu orang yang memiliki keterbatasan motorik \parencite{Kumar2017SolarWheelChair}.

Maka dari itu pada tugas akhir ini diusulkan sistem kendali touchpad berbasis raspberry pi untuk kursi roda elektrik untuk menggantikan sistem kendali standar yaitu joystick. Diharapkan dengan adanya sistem kendali ini dapat memudahkan pasien yang memiliki kebutuhan khusus sehingga dapat mengoperasikan kursi roda dengan mudah.


\section{Rumusan Masalah}

Berdasarkan latar belakang tersebut, kursi roda elektrik masih menggunakan sistem kendali berbentuk joystick, yang mana sistem kendali tersebut masih sulit untuk dioperasikan bagi beberapa orang sehingga diperlukan sistem kendali baru yang mudah untuk digunakan.


\section{Tujuan}

Adapun Tujuan yang ingin dicapai dalam Penelitian ini adalah untuk membuat sistem kendali yang tepat dan mudah untuk digunakan oleh pengguna.


\section{Batasan Masalah}

Penelitian ini berfokus pada pengembangan sistem kendali layar sentuh untuk kursi roda elektrik dengan menggunakan Raspberry Pi sebagai platform utama. Aspek seperti gesture tangan, integrasi perangkat keras, dan daya tanggap sistem akan diprioritaskan. Selain itu, permasalahan seperti masa pakai baterai, integrasi dengan teknologi lain seperti pengenalan suara atau adaptasi terhadap jenis kursi roda tertentu mungkin tidak dibahas secara mendalam dalam Penelitian ini. 


\section{Manfaat}

% Ubah paragraf berikut sesuai dengan tujuan penelitian dari tugas akhir
Manfaat dari keberhasilan pengembangan sistem ini, pengguna kursi roda elektrik akan mendapatkan alat bantu yang memungkinkan mereka bergerak dan menyesuaikan diri dengan lebih mudah. Selain itu, inovasi ini dapat membuka peluang untuk mengintegrasikan teknologi lain, sehingga meningkatkan kemandirian pengguna dan potensi komersial industri terkait. Penelitian ini juga diharapkan dapat menjadi referensi dan inspirasi bagi pengembangan teknologi serupa di masa depan.
